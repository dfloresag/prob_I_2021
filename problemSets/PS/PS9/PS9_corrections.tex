\documentclass[12pt,thmsa]{article}
\usepackage[french,english]{babel}
\usepackage[ansinew]{inputenc}
\usepackage[T1,OT1]{fontenc}
\usepackage{graphicx}
\usepackage{amsmath,amssymb,listings}
\usepackage{alltt,algorithmic,algorithm}
\usepackage{multicol}
\usepackage{cite}
\usepackage{fancyhdr}
\usepackage{setspace}
\usepackage{array}
\usepackage{amsfonts}
\usepackage{latexsym}
\usepackage{epsf}
\usepackage{umlaute}
\usepackage{setspace}
\usepackage{amsthm}
\usepackage{enumerate}


\setlength{\textwidth}{160mm}
\setlength{\textheight}{230mm}
\setlength{\oddsidemargin}{-5mm}
\setlength{\topmargin}{-10mm}

% to get rid of the numbers in the bibliography:
\makeatletter
\def\@biblabel#1{}
\makeatother



\title{Assignement 9}




\begin{document}


\noindent \textsc{University of Geneva}     \hfill \textsc{Bachelor in Economics and Management} \\
\textbf{Probability 1}                      \hfill \textsc{Bachelor in International Relations} \\
Professor Davide La Vecchia                 \hfill Spring 2018  \\
ASSIGNMENT 09                               \hfill  May 3rd \& May 8th



\noindent
\makebox[\linewidth]{\rule{\textwidth}{0.4pt}}\\[1.5ex]


\addtocounter{section}{1}
\section*{Exercise \thesection:}

To solve this exercise we use the table of the cumulative distribution function of the standard normal
{\it i.e.}  $\Phi(\cdot)$, where $P(Z\leq x)=\Phi(x)$.

We use this two properties
\begin{itemize}
\item $P(Z>x)=1-P(Z \leq x)$ (complementarity) and
\item $P(Z\leq-x) = P(Z\geq x)$ (symmetry)
\end{itemize}

\begin{eqnarray*}
P(Z \leq 2.67) &= &\Phi(2.67)  \approx  {0.9962} \\
P(Z\leq 6.2) &=& \Phi(6.2)  \approx  {1.0000}\\
P(Z\leq -1.2) &\stackrel{sym.}{=}& P(Z\geq 1.2) \stackrel{comp.}{=}
1-P(Z < 1.2) = 1-\Phi(1.2) \approx  {0.1151}\\
P(Z\geq 0.68) &\stackrel{comp.}{=}& 1-P(Z < 0.68) = 1-\Phi(0.68)
\approx  {0.2483}\\
P(Z\geq -1.55) &\stackrel{sym.}{=}& P(Z\leq 1.55) = \Phi(1.55) \approx   {0.9394}\\
P(0 \leq Z\leq 0.55) &=& P(Z \leq 0.55) - P(Z < 0) = \Phi(0.55) -
\Phi(0) \approx 0.7088-0.5= {0.2088}\\
P(Z\leq a) & = & 0.9099  \iff  a\approx {1.34}\\
P(Z\leq a) & = & 0.0044  \iff  a\approx {-2.62}\\
P(Z\geq a) & = & 0.7704  \iff  a\approx {-0.74}\\
P(Z\leq a) & = & 0.93  \iff  a\approx {1.48}\\
\end{eqnarray*}


\addtocounter{section}{1}
\section*{Exercise \thesection:}
% ex 3.17

Let $X$ be the random variable measuring the speed. So $X \sim N(72,8^2)$.
\begin{enumerate}
  \item The proportion of drivers who will have to pay a penalty is the probability that $X$ is
  bigger than  80:
\begin{equation*}
P(X>80)= P\left(Z>\frac{80-72}{8} \right)=1-\Phi(1) \simeq 16 \%
\end{equation*}
where $Z \sim N(0,1)$.
  \item We are looking for the conditional probability to measure a speed bigger than 110 given that the driver has already a penalty:
  \begin{eqnarray*}
  P(X>110 \vert X>80) &=&  \frac{P(X>110 \bigcap X>80)}{P(X>80)} \\
   &=& \frac{P(X>110)}{P(X>80)} = \frac{1- \Phi((110-72)/8)}{1-\Phi(1)}\simeq 0.
  \end{eqnarray*}


\end{enumerate}


\addtocounter{section}{1}
\section*{Exercise \thesection:}
% 3.11


Let $X$ (respectively $Y$) be the length of the screws produced by device A (respectively B). The random
variables $X$ and $Y$ are normally distributed with parameters $(8,4)$ and (7.5 ,1).
\begin{enumerate}
  \item We are looking for the device which has the bigger probability to produce screws in the interval $[ 7; 9]$. First:
  \begin{equation*}
  P(7\leq X \leq 9)= P(-0.5\leq Z \leq 0.5)=2 \Phi(0.5)-1 \simeq 0.383
  \end{equation*}
  where $Z$ is the standard normal random variable (i.e. $Z\sim N(0,1)$).
  For the second device:
  \begin{eqnarray*}
  P(7\leq Y \leq 9)&=& P(-0.5\leq Z \leq 1.5)=P(Z \leq 1.5)-P(Z\leq -0.5) \\
  &=&\Phi(1.5)- (1-\Phi(0.5)) \simeq 0.6247
  \end{eqnarray*}
  The device B is the best.

  \item Let $X^*$ be the new random variable describing the length of the screws produced by device A. Given that $E(X^*)=8$ we are looking for the variance such
  that both devices produce the same amount of screws with length in the interval $[7,9]$.
  We want then:
  \begin{equation*}
  P(7\leq X^* \leq 9)= P\left(\frac{-1}{\sigma}\leq Z \leq \frac{1}{\sigma}\right)=2 \Phi\left(\frac{1}{\sigma}\right)-1 = 0.6247
  \end{equation*}
  So we find:
  \begin{equation*}
    \Phi\left(\frac{1}{\sigma}\right)=0.8124  \Leftrightarrow \frac{1}{\sigma}= \Phi^{-1}(0.8124) ,
  \end{equation*}
  and finally:
   \begin{equation*}
   \sigma =\frac{1}{0.89}  \Leftrightarrow \sigma^2 \simeq 1.26.
  \end{equation*}

\end{enumerate}


\addtocounter{section}{1}
\section*{Exercise \thesection:}

The operating time of the machines is given by $M \sim  N(200,81)$ for the machine `Minulta' and by $ C  \sim N(210,144)$ for the machine `Canyon'.
\begin{enumerate}%[(a)]
\item What is the probability that the machine `Canyon' will require maintenance in the next 7 months (assuming 1 month contains 30 days)? Redo the calculation for the machine `Minulta'.

The probability that the machine `Canyon' will need maintenance in the next 7 months (210 days) is $P (C <210)$ whereas for the machine `Minulta', it is equal to $ P ( M <210) $.\\
For the machine `Canyon', we calculate $P(C<210)$:
 \begin{eqnarray*}
P(C<210)&=& \left. P(C-210<0) \right. \nonumber  \\
&=& \left. P \left(\frac{C-210}{\sqrt{144}}<0\right) \right. \nonumber  \\
&=& \left. \Phi(0)=0.5.  \right. \nonumber
 \end{eqnarray*}
 

For the machine `Minulta', we calculate $P(M<210)$:
\begin{eqnarray*}
P(M<210)&=& \left. P(M-200<10) \right. \nonumber  \\
&=& \left. P(\frac{M-200}{\sqrt{81}}<\frac{10}{\sqrt{81}}) \right. \nonumber  \\
&=& \left.P(\frac{M-200}{\sqrt{81}}<\frac{10}{9}) \right. \nonumber  \\
&\approx & \left. \Phi(1.11) \approx 0.8665.  \right. \nonumber
 \end{eqnarray*}

\item What is the probability that the machine `Canyon' will stop before `Minulta' for the first time?

The probability that the machine `Canyon' will stop before `Minulta' for the first time is given by $$P(C<M)=P(C-M<0)$$
We know that $ C $ and $ M $ are independent and normally distributed random variables.

So $$(C-M)\sim N(\mu_c - \mu_m , \sigma_c^2 + \sigma_m^2) \quad
\Rightarrow \quad (C-M)\sim N(10, 225)$$
and
\[
P(C-M<0)=\Phi\left(\frac{0-10}{\sqrt{225}}\right)=\Phi\left(\frac{-10}{15}\right)\approx\Phi(-0.67) = 1-\Phi(0.67) \approx 1 - 0.7486 = 0.2514.
\]
\item What is the probability that the machine `Minolta' will stop before `Canyon' for the first time?

The probability that the machine `Minolta' will stop before `Canyon' for the first time is given by $$(M-C)\sim N(-10,225)$$ 
So
\[
P(M-C<0)=\Phi\left(\frac{0-(-10)}{\sqrt{225}}\right)=\Phi\left(\frac{10}{15}\right)\approx\Phi(0.67) \approx 0.7486.
\]

\medskip

Alternative: Note that: $P(M - C < 0)=1-P(C-M<0)$.
\item What is the probability that both machines will stop at the same time for the first time?

The probability that both machines will stop at the same time for the first time is: $$P(C=M)=P(C-M=0).$$

The probability that a continuous random variable has an exact value is zero, hence: $$P(C=M)=0.$$



\end{enumerate}



\end{document}



