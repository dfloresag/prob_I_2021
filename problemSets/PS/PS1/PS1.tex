\documentclass[12pt,a4paper,titlepage]{article}\usepackage[]{graphicx}\usepackage[]{color}
% maxwidth is the original width if it is less than linewidth
% otherwise use linewidth (to make sure the graphics do not exceed the margin)
\makeatletter
\def\maxwidth{ %
  \ifdim\Gin@nat@width>\linewidth
    \linewidth
  \else
    \Gin@nat@width
  \fi
}
\makeatother

\definecolor{fgcolor}{rgb}{0.345, 0.345, 0.345}
\newcommand{\hlnum}[1]{\textcolor[rgb]{0.686,0.059,0.569}{#1}}%
\newcommand{\hlstr}[1]{\textcolor[rgb]{0.192,0.494,0.8}{#1}}%
\newcommand{\hlcom}[1]{\textcolor[rgb]{0.678,0.584,0.686}{\textit{#1}}}%
\newcommand{\hlopt}[1]{\textcolor[rgb]{0,0,0}{#1}}%
\newcommand{\hlstd}[1]{\textcolor[rgb]{0.345,0.345,0.345}{#1}}%
\newcommand{\hlkwa}[1]{\textcolor[rgb]{0.161,0.373,0.58}{\textbf{#1}}}%
\newcommand{\hlkwb}[1]{\textcolor[rgb]{0.69,0.353,0.396}{#1}}%
\newcommand{\hlkwc}[1]{\textcolor[rgb]{0.333,0.667,0.333}{#1}}%
\newcommand{\hlkwd}[1]{\textcolor[rgb]{0.737,0.353,0.396}{\textbf{#1}}}%
\let\hlipl\hlkwb

\usepackage{framed}
\makeatletter
\newenvironment{kframe}{%
 \def\at@end@of@kframe{}%
 \ifinner\ifhmode%
  \def\at@end@of@kframe{\end{minipage}}%
  \begin{minipage}{\columnwidth}%
 \fi\fi%
 \def\FrameCommand##1{\hskip\@totalleftmargin \hskip-\fboxsep
 \colorbox{shadecolor}{##1}\hskip-\fboxsep
     % There is no \\@totalrightmargin, so:
     \hskip-\linewidth \hskip-\@totalleftmargin \hskip\columnwidth}%
 \MakeFramed {\advance\hsize-\width
   \@totalleftmargin\z@ \linewidth\hsize
   \@setminipage}}%
 {\par\unskip\endMakeFramed%
 \at@end@of@kframe}
\makeatother

\definecolor{shadecolor}{rgb}{.97, .97, .97}
\definecolor{messagecolor}{rgb}{0, 0, 0}
\definecolor{warningcolor}{rgb}{1, 0, 1}
\definecolor{errorcolor}{rgb}{1, 0, 0}
\newenvironment{knitrout}{}{} % an empty environment to be redefined in TeX

\usepackage{alltt}
\usepackage[french,english]{babel}
\usepackage[latin1]{inputenc}
\usepackage[T1,OT1]{fontenc}
\usepackage{graphicx}
\usepackage{amsmath,amssymb,listings}
\usepackage{alltt,algorithmic,algorithm}
\usepackage{multicol}
\usepackage{cite}
\usepackage{fancyhdr}

\setlength{\textwidth}{160mm}
\setlength{\textheight}{230mm}
\setlength{\oddsidemargin}{-5mm}
\setlength{\topmargin}{-10mm}

% to get rid of the numbers in the bibliography:
\makeatletter
\def\@biblabel#1{}
\makeatother

%\pagestyle{fancy}
%\fancyhf{}
%\lhead{UNIVERSITY OF GENEVA }
%\rhead{MASTER OF SCIENCE IN ECONOMICS}



\title{Assignement 1}
\IfFileExists{upquote.sty}{\usepackage{upquote}}{}
\begin{document}


\noindent \textsc{University of Geneva}     \hfill \textsc{Bachelor in Economics and Management} \\
\textbf{Probability 1}                      \hfill \textsc{Bachelor in International Relations} \\
Dr. Daniel \textsc{Flores Agreda}                 \hfill Spring 2021  \\
ASSIGNMENT 01                               \hfill

%\begin{center}

%\vspace{2cm}
%\LARGE{\textbf{Assignement 1}}
%\end{center}

\noindent
\makebox[\linewidth]{\rule{\textwidth}{0.4pt}}\\[1.5ex]

%\vspace{1cm}




\section{Integrals}


Compute the following integrals: \\
\begin{align*}
\int& 5x^{3}-10x^{-6}+4 dx \\
\int&  \pi^{8}+\pi^{-8} dx \\
\int& 3\sqrt[4]{x^{3}}+\frac{7}{x^{5}}+\frac{1}{6\sqrt{x}}dx \\
\int& \frac{15}{x}dx \\
\int & exp(x)dx\\
\int_{1}^{2}& y^{2}+y^{-2}dy \\
\int_{-3}^{1}& 6y^{2}-5y+2dy \\
\int_{0}^{u}& \lambda exp(\lambda y) dy \\
\int_{-\infty}^{a}& \frac{1}{2b} exp\left (-\frac{|y-\mu|}{b}\right) dy, \quad \text{for} \quad \mu \in \mathbb{R} \quad \text{and} \quad b \in \mathbb{R}^{+}.
\end{align*}


\vspace{1cm}


\section{Combinatorics}

Combinatorics is a branch of mathematics concerning the study of finite or countable discrete structures.
We will enumerate the number of ways of selecting $k$ items from a collection of $n$ objects.
In this class we will use several kinds of combinations:

\begin{enumerate}
  \item Permutation with repetition: $n^k$  \\
  we draw (with replacement) $k$ elements from $n$ objects.
  \item Permutation without repetition: $P^n_n=n!$ \\
  we draw (without replacement) $n$ elements from $n$ objects.
  \item Partial permutations: $P^n_k=\frac{n!}{(n-k)!}$ \\
  we draw (without replacement) $k$ elements from $n$ objects.
  \item Combinations: $C{n \choose k}=\frac{n!}{k!(n-k)!}$ \\
  Same as a partial permutation but the order of the items doesn't count.
  The formula comes from:
  $$ \text{Partial permutations}= \text{Permutations of the outcome} \times \text{Combinations}$$
$$ \text{That is :} \quad P^n_k=P^k_k \times {n \choose k} $$
\end{enumerate}

\textbf{Theorem:}
\begin{align}
{n \choose r}={n-1 \choose r-1}+{n-1 \choose r} \quad \text{for } 1\leq r\leq n
\end{align}

\vspace{1cm}

1) Assume there are 6 women and 4 men taking an exam. Then they are ranked according to their grades.
\begin{itemize}
  \item How many possible rankings can we get?
  \item If the women are ranked between them, and the men between them. How many possible global rankings do we have?
\end{itemize}

2) How many car number plates with 7 items does it exist:
\begin{itemize}
  \item if the first two are letters and the five last are digits?
  \item same question but we assume that there is no repetition of the letters and the digits.
\end{itemize}

3) Suppose that a fair coin is tossed five times. The outcome of the experiment is registered with $ H $ if the result of the tossing is head and with $ T $ if the outcome is tail (the outcomes would look like $ HHTHT, THTTH $ and so on).
\begin{itemize}
  \item How many possible outcomes do we have?
  \item How many of these five dimensional outcomes would include two heads?
\end{itemize}

4) A committee of 3 is to be formed from a group of 20 people. How many different committees are possible?

\vspace{1cm}



\section{Taylor Expansions}


\textbf{Binomial Theorem:}
\begin{align}
(a+b)^{n}&=\sum_{k=0}^{n}{n \choose k} a^{k} b^{n-k}
\end{align}

A Taylor series of a function $ f(x) $ about $ x_0 $ is the power series
\begin{eqnarray}
f(x)&=&f(x_0)+f'(x_0)(x-x_0)+\frac{f''(x_0)}{2!}(x-x_0)^{2}+...+\frac{f^{(n)}(x_0)}{n!}(x-x_0)^{n}+... \\
&=& \sum_{k=0}^\infty \frac{f^{(k)}(x_0)}{k!}(x-x_0)^{k}
\end{eqnarray}
A Maclaurin series is a Taylor expansion of a function about $ x_0=0 $.\newline

1) Using binomial theorem, compute the following expansions:
\begin{align*}
 &(x+y)^{2}  \\
 &(x-y)^{2}\\
 &x^{2}-y^{2}\\
 &(x+y)^{3} \\
 &x^{3}+y^{3}\\
\end{align*}

2) Find the Maclaurin expansions of the following functions
$ f(x)=\frac{1}{1-x} $ and $ g(x)=exp(x)$.


\vspace{1cm}



\section{Sums}

1) Compute the sum of the first 100 integers, namely find the value of
$$
\sum_{i=1}^{100} i =1 +2 + 3+ .....+100.
$$

2) Find a general formula for: $\sum_{i=1}^n i$.


\end{document}
