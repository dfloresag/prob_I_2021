\documentclass[12pt,thmsa]{article}\usepackage[]{graphicx}\usepackage[]{color}
% maxwidth is the original width if it is less than linewidth
% otherwise use linewidth (to make sure the graphics do not exceed the margin)
\makeatletter
\def\maxwidth{ %
  \ifdim\Gin@nat@width>\linewidth
    \linewidth
  \else
    \Gin@nat@width
  \fi
}
\makeatother

\definecolor{fgcolor}{rgb}{0.345, 0.345, 0.345}
\newcommand{\hlnum}[1]{\textcolor[rgb]{0.686,0.059,0.569}{#1}}%
\newcommand{\hlstr}[1]{\textcolor[rgb]{0.192,0.494,0.8}{#1}}%
\newcommand{\hlcom}[1]{\textcolor[rgb]{0.678,0.584,0.686}{\textit{#1}}}%
\newcommand{\hlopt}[1]{\textcolor[rgb]{0,0,0}{#1}}%
\newcommand{\hlstd}[1]{\textcolor[rgb]{0.345,0.345,0.345}{#1}}%
\newcommand{\hlkwa}[1]{\textcolor[rgb]{0.161,0.373,0.58}{\textbf{#1}}}%
\newcommand{\hlkwb}[1]{\textcolor[rgb]{0.69,0.353,0.396}{#1}}%
\newcommand{\hlkwc}[1]{\textcolor[rgb]{0.333,0.667,0.333}{#1}}%
\newcommand{\hlkwd}[1]{\textcolor[rgb]{0.737,0.353,0.396}{\textbf{#1}}}%
\let\hlipl\hlkwb

\usepackage{framed}
\makeatletter
\newenvironment{kframe}{%
 \def\at@end@of@kframe{}%
 \ifinner\ifhmode%
  \def\at@end@of@kframe{\end{minipage}}%
  \begin{minipage}{\columnwidth}%
 \fi\fi%
 \def\FrameCommand##1{\hskip\@totalleftmargin \hskip-\fboxsep
 \colorbox{shadecolor}{##1}\hskip-\fboxsep
     % There is no \\@totalrightmargin, so:
     \hskip-\linewidth \hskip-\@totalleftmargin \hskip\columnwidth}%
 \MakeFramed {\advance\hsize-\width
   \@totalleftmargin\z@ \linewidth\hsize
   \@setminipage}}%
 {\par\unskip\endMakeFramed%
 \at@end@of@kframe}
\makeatother

\definecolor{shadecolor}{rgb}{.97, .97, .97}
\definecolor{messagecolor}{rgb}{0, 0, 0}
\definecolor{warningcolor}{rgb}{1, 0, 1}
\definecolor{errorcolor}{rgb}{1, 0, 0}
\newenvironment{knitrout}{}{} % an empty environment to be redefined in TeX

\usepackage{alltt}
\usepackage[french,english]{babel}
\usepackage[ansinew]{inputenc}
\usepackage[T1,OT1]{fontenc}
\usepackage{graphicx}
\usepackage{amsmath,amssymb,listings}
\usepackage{alltt,algorithmic,algorithm}
\usepackage{multicol}
\usepackage{cite}
\usepackage{fancyhdr}
\usepackage{setspace}
\usepackage{array}
\usepackage{amsfonts}
\usepackage{latexsym}
\usepackage{epsf}
\usepackage{umlaute}
\usepackage{setspace}
\usepackage{amsthm}
\usepackage{enumerate}


\setlength{\textwidth}{160mm}
\setlength{\textheight}{230mm}
\setlength{\oddsidemargin}{-5mm}
\setlength{\topmargin}{-10mm}

% to get rid of the numbers in the bibliography:
\makeatletter
\def\@biblabel#1{}
\makeatother



\title{Assignement 5}
\IfFileExists{upquote.sty}{\usepackage{upquote}}{}
\begin{document}


\noindent \textsc{University of Geneva}     \hfill \textsc{Bachelor in Economics and Management} \\
\textbf{Probability 1}                      \hfill \textsc{Bachelor in International Relations} \\
Dr. Daniel \textsc{Flores Agreda}                 \hfill Spring 2021  \\
ASSIGNMENT 05



\noindent
\makebox[\linewidth]{\rule{\textwidth}{0.4pt}}\\[1.5ex]

\section*{Exercise 1}

You look at the following discrete probability distributions for two random variables X and Y respectively
\begin{center}
\begin{tabular}{l*{6}{c}r}
X \text{values}               & -1 & 1 & 3 & 5 & 7 \\
\hline
\text{Probability}         & 0.1 & 0.2 & 0.4 & 0.2 & 0.1  \\
\end{tabular}
\end{center}
\begin{center}
\begin{tabular}{l*{6}{c}r}
Y \text{values}               & -1 & 1 & 3 & 5 & 7 \\
\hline
\text{Probability}         & 0.2 & 0.15 & 0.3 & 0.15 & 0.2  \\
\end{tabular}
\end{center}

Compute the mean and the variance of both variables.


\section*{Exercise 2}

In a TV game, a candidate faces 5 doors, one of which hides a gift. Viewers can make bets on the number of doors that the candidate will push until he finds the gift. Jules and Gaston are candidates for the game. Jules has a good memory and is not likely to push twice the same door. As far as Gaston is concerned, he has absolutely no memory.

\begin{enumerate}%[(a)]
\item Construct the probability function of the number of doors pushed by Jules. Calculate the expectation of this event.
\item Construct the probability function of the number of doors pushed by Gaston. Calculate the expectation of this event.



\noindent \underline{Indication:} The following mathematical property can be used (generic geometric series):
$$|x|<1 \Rightarrow \sum_{k=1}^{\infty} k x^{k-1} = \frac{1}{(1-x)^2}.$$
\item Generalize the previous results to n doors.
\end{enumerate}

\newpage
\section*{Exercise 3} %\bf Exercice 4.7 du recueil d'exercices}

Two cards are randomly chosen from a box containing the following five cards: 1, 1, 2, 2, 3. $X$ represents the average and $Y$ is the maximum of the two numbers drawn.
\begin{enumerate}%[(a)]
    \item Calculate the probability distribution, the cumulative distribution function, the mean and the variance of $X$, $Y$ and $Z = Y - X$.
    \item If $W$ is the sum of the two numbers, what are its expectation and its variance?
\end{enumerate}


\section*{Exercise 4} %\bf}

We consider a die whose faces are numbered from 1 to 6 and we define $X$ the random variable given by the number of the upper face. It is assumed that the die is rigged so that the probability of getting a face is proportional to the number on that face.

 \begin{enumerate}%[(a)]
\item Determine the probability function of $X$, then calculate its expectation and variance.
\item We define $Y = 3X$. Calculate the expectation and variance of $Y$. Is it necessary to determine the probability function of $Y$ for calculating its expectation and variance?
\item We define $Z = \frac{1}{X}$. Calculate the expectation and variance of $Z$. Is it necessary to determine the probability function of $Z$ for calculating its expectation and variance?
\end{enumerate}


\section*{Exercise 5} %\bf}

\begin{enumerate}
\item The following game is considered: The player rolls once a fair six sided die. If he gets 1, 2 or 3, he wins the equivalent in francs. Otherwise, he loses 2 francs. Let $ X $ be the random variable corresponding to the player's win (a negative value indicating a loss).

\begin{enumerate}%[(a)]
\item Give the probability function of $X$ and its distribution function $F_X$.
\item Calculate the expectation and variance of $X$.
\end{enumerate}

\item We modify the game as follows: The winnings remain the same for the results 1, 2 or 3, but if the player gets something else, he rolls the die again. If he then gets 3 or less, he wins 3 francs, otherwise he loses 5 francs. We define $Y$ as the random variable corresponding to the gain of the player in this new game.

\begin{enumerate}%[(a)]
\item Give the probability function of $Y$ and its distribution function $F_Y$.
\item Calculate the expectation and variance of $Y$.
\end{enumerate}

\item Which game is the most advantageous for the player? To justify.

\end{enumerate}



\section*{Exercise 6 (Optional)}
The random variable $ X $ is Bernoulli(p) distribution if its probability mass function is given by:
\begin{align*}
P(X=x)=p^{x}(1-p)^{1-x} & \text{for } x=0,1
\end{align*}
where $ 0<p<1 $.
Compute the Mean and the Variance of the Bernoulli distribution.



\section*{Exercise 7 (Optional)}

Let $X$ a discrete random variable following a Poisson distribution with parameter $\lambda$. Its probability function is given by:
\begin{equation*}
P(X=k)=e^{-\lambda} \frac{\lambda^{k}}{k!} \quad \text{for} \quad k=0,1,...
\end{equation*}
\begin{enumerate}
  \item Check that $P(X=k)$ is a probability function.
  \item Prove that its expectation is equal to $\lambda$.
  \item Find the variance of $X$.
  \item Compare the expectation and the variance of $X$ with the expectation and the variance of $S$, for a very large $n$. Here $S \sim \mathcal{B}(n,p)$.
\end{enumerate}







\end{document}



