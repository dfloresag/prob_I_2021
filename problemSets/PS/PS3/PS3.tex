\documentclass[12pt,thmsa]{article}\usepackage[]{graphicx}\usepackage[]{color}
% maxwidth is the original width if it is less than linewidth
% otherwise use linewidth (to make sure the graphics do not exceed the margin)
\makeatletter
\def\maxwidth{ %
  \ifdim\Gin@nat@width>\linewidth
    \linewidth
  \else
    \Gin@nat@width
  \fi
}
\makeatother

\definecolor{fgcolor}{rgb}{0.345, 0.345, 0.345}
\newcommand{\hlnum}[1]{\textcolor[rgb]{0.686,0.059,0.569}{#1}}%
\newcommand{\hlstr}[1]{\textcolor[rgb]{0.192,0.494,0.8}{#1}}%
\newcommand{\hlcom}[1]{\textcolor[rgb]{0.678,0.584,0.686}{\textit{#1}}}%
\newcommand{\hlopt}[1]{\textcolor[rgb]{0,0,0}{#1}}%
\newcommand{\hlstd}[1]{\textcolor[rgb]{0.345,0.345,0.345}{#1}}%
\newcommand{\hlkwa}[1]{\textcolor[rgb]{0.161,0.373,0.58}{\textbf{#1}}}%
\newcommand{\hlkwb}[1]{\textcolor[rgb]{0.69,0.353,0.396}{#1}}%
\newcommand{\hlkwc}[1]{\textcolor[rgb]{0.333,0.667,0.333}{#1}}%
\newcommand{\hlkwd}[1]{\textcolor[rgb]{0.737,0.353,0.396}{\textbf{#1}}}%
\let\hlipl\hlkwb

\usepackage{framed}
\makeatletter
\newenvironment{kframe}{%
 \def\at@end@of@kframe{}%
 \ifinner\ifhmode%
  \def\at@end@of@kframe{\end{minipage}}%
  \begin{minipage}{\columnwidth}%
 \fi\fi%
 \def\FrameCommand##1{\hskip\@totalleftmargin \hskip-\fboxsep
 \colorbox{shadecolor}{##1}\hskip-\fboxsep
     % There is no \\@totalrightmargin, so:
     \hskip-\linewidth \hskip-\@totalleftmargin \hskip\columnwidth}%
 \MakeFramed {\advance\hsize-\width
   \@totalleftmargin\z@ \linewidth\hsize
   \@setminipage}}%
 {\par\unskip\endMakeFramed%
 \at@end@of@kframe}
\makeatother

\definecolor{shadecolor}{rgb}{.97, .97, .97}
\definecolor{messagecolor}{rgb}{0, 0, 0}
\definecolor{warningcolor}{rgb}{1, 0, 1}
\definecolor{errorcolor}{rgb}{1, 0, 0}
\newenvironment{knitrout}{}{} % an empty environment to be redefined in TeX

\usepackage{alltt}
\usepackage[french,english]{babel}
\usepackage[ansinew]{inputenc}
\usepackage[T1,OT1]{fontenc}
\usepackage{graphicx}
\usepackage{amsmath,amssymb,listings}
\usepackage{alltt,algorithmic,algorithm}
\usepackage{multicol}
\usepackage{cite}
\usepackage{fancyhdr}
\usepackage{setspace}
\usepackage{array}
\usepackage{amsfonts}
\usepackage{latexsym}
\usepackage{epsf}
\usepackage{umlaute}
\usepackage{setspace}
\usepackage{amsthm}
\usepackage{enumerate}


\setlength{\textwidth}{160mm}
\setlength{\textheight}{230mm}
\setlength{\oddsidemargin}{-5mm}
\setlength{\topmargin}{-10mm}

% to get rid of the numbers in the bibliography:
\makeatletter
\def\@biblabel#1{}
\makeatother



\title{Assignement 3}
\IfFileExists{upquote.sty}{\usepackage{upquote}}{}
\begin{document}


\noindent \textsc{University of Geneva}     \hfill \textsc{Bachelor in Economics and Management} \\
\textbf{Probability 1}                      \hfill \textsc{Bachelor in International Relations} \\
Dr. Daniel \textsc{Flores Agreda}                 \hfill Spring 2021  \\
ASSIGNMENT 03



\noindent
\makebox[\linewidth]{\rule{\textwidth}{0.4pt}}\\[1.5ex]

\section*{Exercise 1}


The mail order company `CD-Bill' sends a questionnaire to households
in its address file to know if they are interested in rap,
classical or rock  music, genres in which it is specialized. By searching the answers, it
notes that 20\% of households are not interested in any of these 3 genres of music, 35\% of households are interested in rap, 20\% in classical music and the number of households interested in rock is twice as many as in classical music. In addition, 5\%
households are interested in both rap and rock and no household is interested
in both rap and classical music.

\medskip

\noindent If one randomly chooses one of the households that responded to the
  survey:

 \begin{enumerate}%[(a)]
 \item What is the probability that he/she is interested in rap or
classical music?


 \item What is the probability that he/she is interested in rock, knowing
that he/she is interested in rap?


 \item What is the probability that he/she is interested in at least 2 kinds of
music ?

 \item What is the probability that he/she is interested in rap, knowing that he/she is interested in neither classical nor rock music?
 \end{enumerate}


\section*{Exercise 2}



Roger Federer prepares for Wimbledon tennis. Since turf is the best surface, we know that regardless of the player he faces, the probability of winning a match is 0.85. To win the tournament a player has to win seven consecutive games.

\begin{enumerate} %[(a)]

\item What is the probability that Federer reaches at least the semi-finals?

\item Federer reached the quarter-finals (5th round). What is the probability that he wins the tournament?

\item A journalist says that if Novak Djokovic is in the final with Federer, the latter has a 60\% chance of winning the tournament against 90\% if Djokovic does not qualify.
 He says Djokovic has a 75\% chance of reaching the final. Calculate the probability that Federer\footnote{Assuming he is in the final.} wins the tournament according to the  journalist.

\end{enumerate}

\section*{Exercise 3}

The probability that a new car battery works for over  30'000km is
0.8, the probability that it functions for over 60'000km is 0.4, and the probability that it
works for over 90'000km is 0.1. If a new car battery is still working after 30'000km,
what is the probability that

\begin{enumerate}
\item  its total life will exceed 60'000km?
\item its additional life will exceed 60'000km?

\end{enumerate}


\section*{Exercise 4}
Consider the random experiment of tossing a fair coin twice. Let us define the following events:
\begin{itemize}
\item $ A $: Observe a head $ (H) $ on the first toss
\item $ B $: Observe a head $ (H) $ on the second toss
\item $ C $: Observe the same outcome on both tosses
\end{itemize}
Are the events pairwise independent? Are the events jointly independent?


\section*{Exercise 5 (Optional)}


Given the axioms of probability theory, show the following:
\begin{enumerate}
  \item $P(\emptyset)=0$
  \item $P(A^c)=1-P(A)$
  \item $P(A \cap B) \leq \min(P(A),P(B))$
  \item $P(A \cup B)=P(A)+P(B)-P(A \cap B)$
  \item If $A_1, A_2,...$ are mutually exclusive events in $\mathcal{B}$, then
\begin{equation}
P\left(  \bigcup_{i=1}^{n} A_i \right) = \sum_{i=1}^{n} P(A_i).  \label{Eq: ProbUnionDisj}
\end{equation}
  \item Let's $A_n$ a collection of sets such that: $\lim_{n\rightarrow \infty}A_n= A$.
  Show that $\lim_{n\rightarrow \infty}P(A_n)=P(A)$.
\end{enumerate}













\end{document}




