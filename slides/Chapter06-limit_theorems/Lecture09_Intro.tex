\documentclass[notes=show,smaller,handout]{beamer}\usepackage[]{graphicx}\usepackage[]{color}
% maxwidth is the original width if it is less than linewidth
% otherwise use linewidth (to make sure the graphics do not exceed the margin)
\makeatletter
\def\maxwidth{ %
  \ifdim\Gin@nat@width>\linewidth
    \linewidth
  \else
    \Gin@nat@width
  \fi
}
\makeatother

\definecolor{fgcolor}{rgb}{0.345, 0.345, 0.345}
\newcommand{\hlnum}[1]{\textcolor[rgb]{0.686,0.059,0.569}{#1}}%
\newcommand{\hlstr}[1]{\textcolor[rgb]{0.192,0.494,0.8}{#1}}%
\newcommand{\hlcom}[1]{\textcolor[rgb]{0.678,0.584,0.686}{\textit{#1}}}%
\newcommand{\hlopt}[1]{\textcolor[rgb]{0,0,0}{#1}}%
\newcommand{\hlstd}[1]{\textcolor[rgb]{0.345,0.345,0.345}{#1}}%
\newcommand{\hlkwa}[1]{\textcolor[rgb]{0.161,0.373,0.58}{\textbf{#1}}}%
\newcommand{\hlkwb}[1]{\textcolor[rgb]{0.69,0.353,0.396}{#1}}%
\newcommand{\hlkwc}[1]{\textcolor[rgb]{0.333,0.667,0.333}{#1}}%
\newcommand{\hlkwd}[1]{\textcolor[rgb]{0.737,0.353,0.396}{\textbf{#1}}}%
\let\hlipl\hlkwb

\usepackage{framed}
\makeatletter
\newenvironment{kframe}{%
 \def\at@end@of@kframe{}%
 \ifinner\ifhmode%
  \def\at@end@of@kframe{\end{minipage}}%
  \begin{minipage}{\columnwidth}%
 \fi\fi%
 \def\FrameCommand##1{\hskip\@totalleftmargin \hskip-\fboxsep
 \colorbox{shadecolor}{##1}\hskip-\fboxsep
     % There is no \\@totalrightmargin, so:
     \hskip-\linewidth \hskip-\@totalleftmargin \hskip\columnwidth}%
 \MakeFramed {\advance\hsize-\width
   \@totalleftmargin\z@ \linewidth\hsize
   \@setminipage}}%
 {\par\unskip\endMakeFramed%
 \at@end@of@kframe}
\makeatother

\definecolor{shadecolor}{rgb}{.97, .97, .97}
\definecolor{messagecolor}{rgb}{0, 0, 0}
\definecolor{warningcolor}{rgb}{1, 0, 1}
\definecolor{errorcolor}{rgb}{1, 0, 0}
\newenvironment{knitrout}{}{} % an empty environment to be redefined in TeX

\usepackage{alltt}
%%%%%%%%%%%%%%%%%%%%%%%%%%%%%%%%%%%%%%%%%%%%%%%%%%%%%%%%%%%%%%%%%%%%%%%%%%%%%%%%%%%%%%%%%%%%%%%%%%%%%%%%%%%%%%%%%%%%%%%%%%%%%%%%%%%%%%%%%%%%%%%%%%%%%%%%%%%%%%%%%%%%%%%%%%%%%%%%%%%%%%%%%%%%%%%%%%%%%%%%%%%%%%%%%%%%%%%%%%%%%%%%%%%%%%%%%%%%%%%%%%%%%%%%%%%%
\usepackage{amssymb}
\usepackage{amsmath}
\usepackage{graphicx}
\usepackage{hyperref}
\usepackage{multimedia}
\usepackage{epstopdf}
\usepackage{color}

\setcounter{MaxMatrixCols}{10}
\newtheorem{remark}{Remark}[section]
\newtheorem{proposition}{Proposition}[section]
\newtheorem{interpretation}{Interpretation}[section]
\newtheorem{goal}{Goal}[section]
\newtheorem{statement}{Statement}[section]
\newtheorem{aes}{Aim \& Scope}[section]
\newtheorem{exercise}{Exercise}[section]
\renewcommand{\Pr}{P}

\newcommand{\mbf}[1]{\mathbf{#1}}
\newcommand{\beq}{\begin{equation}}
\newcommand{\eeq}{\end{equation}}
\newcommand{\bea}{\begin{eqnarray}}
\newcommand{\eea}{\end{eqnarray}}
\newcommand{\ba}{\begin{array}}
\newcommand{\ea}{\end{array}}
\newcommand{\bi}{\begin{itemize}}
\newcommand{\ei}{\end{itemize}}
\newcommand{\ben}{\begin{enumerate}}
\newcommand{\een}{\end{enumerate}}
\newcommand{\nn}{\nonumber}

\newenvironment{stepenumerate}{\begin{enumerate}[<+->]}{\end{enumerate}}
\newenvironment{stepitemize}{\begin{itemize}[<+->]}{\end{itemize} }
\newenvironment{stepenumeratewithalert}{\begin{enumerate}[<+-| alert@+>]}{\end{enumerate}}
\newenvironment{stepitemizewithalert}{\begin{itemize}[<+-| alert@+>]}{\end{itemize} }
\usetheme{Madrid}


%%%%%%%%%%%%%%%%%%%%%%%%%%%%%%%%%%%%%%%%%%%%%%%%%%%%%%%%%%%%%%%%%%%%%%%%%%%%%%%
% GSEM COLORS
%%%%%%%%%%%%%%%%%%%%%%%%%%%%%%%%%%%%%%%%%%%%%%%%%%%%%%%%%%%%%%%%%%%%%%%%%%%%%%%
\definecolor{darkGSEM}{RGB}{70,95,127}
\definecolor{darkGSEM2}{RGB}{40,80,150}
\definecolor{GSEM}{RGB}{96,121,153} % GSEM 10% lighter

%%% Global colors
\setbeamercolor*{palette primary}{use=structure,fg=white,bg=darkGSEM}
\setbeamercolor*{palette quaternary}{use=structure,fg=white,bg=darkGSEM!90}
\setbeamercolor{frametitle}{fg=white,bg=GSEM!80}

%%% TOC colors
\setbeamercolor{section in toc}{fg=darkGSEM}

%%% itemize colors
\setbeamertemplate{itemize items}[circle]
\setbeamercolor{itemize item}{fg=darkGSEM2}
\setbeamercolor{itemize subitem}{fg=darkGSEM2}
\setbeamercolor{itemize subsubitem}{fg=darkGSEM2}


%%% enumerate colors
\setbeamercolor{item projected}{fg=white,bg=GSEM}
\setbeamertemplate{enumerate item}{\insertenumlabel.}
\setbeamercolor{enumerate item}{fg=darkGSEM2}
\setbeamercolor{enumerate subitem}{fg=darkGSEM2}
\setbeamercolor{enumerate subsubitem}{fg=darkGSEM2}


\AtBeginSection[]
{
  \begin{frame}{\secname}{\secname}
    \frametitle{Outline}
    \tableofcontents[currentsection]
  \end{frame}
}

%%%%%%%%%%%%%%%%%%%%%%%%%%%%%%%%%%%%%%%%%%%%%%%%%%%%%%%%%%%%%%%%%%%%%%%%%%%%%%%%
% R Chunk Options and Packages
%%%%%%%%%%%%%%%%%%%%%%%%%%%%%%%%%%%%%%%%%%%%%%%%%%%%%%%%%%%%%%%%%%%%%%%%%%%%%%%%




%%%%%%%%%%%%%%%%%%%%%%%%%%%%%%%%%%%%%%%%%%%%%%%%%%%%%%%%%%%%%%%%%%%%%%%%%%%%%%%%
\title[S110015]{Probability 1}
\subtitle{Chapter 09 : Reminder - Sequences and Series}
\author[Flores-Agreda, La Vecchia]{Dr. Daniel Flores-Agreda, \\[0.5em] \tiny{(based on the notes of Prof. Davide La Vecchia)}}
\date{Spring Semester 2021}
\IfFileExists{upquote.sty}{\usepackage{upquote}}{}
\begin{document}

\begin{frame}{Outline}
  \titlepage
\end{frame}


% \begin{frame}{Objectives}
%   \begin{itemize}
%   \item . \bigskip
%   \item .
%   \end{itemize}
% \end{frame}

\begin{frame}{\secname}
  \frametitle{Outline}
  \tableofcontents
\end{frame}

%%%%%%%%%%%%%%%%%%%%%%%%%%%%%%%%%%%%%%%%%%%%%%%%%%%%%%%%%%%%%%%%%%%%%%%%%%%%%%%%
\section{Sequences of real numbers }
%%%%%%%%%%%%%%%%%%%%%%%%%%%%%%%%%%%%%%%%%%%%%%%%%%%%%%%%%%%%%%%%%%%%%%%%%%%%%%%%

\begin{frame}{\secname}

\frametitle{Sequences of real numbers }

\begin{definition}
 A sequence is an ordered list of real numbers
of the form%
\begin{equation*}
a_{1},a\,_{2},...,a_{n},...
\end{equation*}
where each natural number $n \in \mathbb{N}$ corresponds exactly to a real number $a_n \in \mathbb{R}$. A sequence is denoted
by $\{a_n \}_{n \in \mathbb{N}}$, where $n$ is called the index of the sequence and $a_n$ is its n-th term.
\end{definition}

\vspace{0.4cm}

Remark: the sequence can contain infinite terms...

\begin{example}
The list of numbers
$$
\Big\{1,\frac{1}{2},\frac{1}{3},\frac{1}{4},... \Big \}
$$
is a sequence, where each natural number corresponds the real number $a_n = \frac{1}{n}$.
\end{example}



\end{frame}

%%%%%%%%%%%%%%%%%%%%%%%%%%%%%%%%%%%%%%%%%%%%%%%%%%%%%%%%%%%%%%%%%%%%%%%%%%%%%%%%
\section{Limit of a sequences of real numbers }
%%%%%%%%%%%%%%%%%%%%%%%%%%%%%%%%%%%%%%%%%%%%%%%%%%%%%%%%%%%%%%%%%%%%%%%%%%%%%%%%


\begin{frame}{\secname}



\begin{definition}
A real number  $a\in \mathbb{R}$ (a is a finite number) is called the limit of a sequence $\{a_n \}_{n \in \mathbb{N}}$ if, for any $\epsilon >0$, a natural number $n(\epsilon) \in \mathbb{N}$ exists such that
\begin{equation} \label{Eq. lim}
\vert a_{n}-a \vert < \epsilon \quad  \text{for all \ \ } n\geq n(\epsilon).
\end{equation}
If for a given sequence $\{a_n \}_{n \in \mathbb{N}}$ the real number $a$ satisfies (\ref{Eq. lim}), then we write
$$
a = \lim_{n\to \infty} a_n.
$$
\end{definition}

\begin{example}
The sequence $\{ a_n \}_{n \in \mathbb{N}}$
with $a_n = \frac{1}{n}$, converges to zero.
\end{example}

Remark: loosely speaking,  Eq. (\ref{Eq. lim}) states that, for $n \geq n(\epsilon)$, $a_n$ is \textbf{always} close to $a$ (or equivalently, the difference in absolute value between $a_n$ and $a$ is \textbf{never} large).

\end{frame}%


%%%%%%%%%%%%%%%%%%%%%%%%%%%%%%%%%%%%%%%%%%%%%%%%%%%%%%%%%%%%%%%%%%%%%%%%%%%%%%%%
\section{Series}
%%%%%%%%%%%%%%%%%%%%%%%%%%%%%%%%%%%%%%%%%%%%%%%%%%%%%%%%%%%%%%%%%%%%%%%%%%%%%%%%

\begin{frame}{\secname}

\begin{definition}
Let $\{ a_k \}_{k \in \mathbb{N}}$ be a  sequence. The sum of the first $n$ terms of  $\{ a_k \}_{k \in \mathbb{N}}$:
$$
s_n = \sum_{k=1}^{n} a_k = a_1 + a_2 +...+a_n
$$
is called the $n$-th partial sum of $\{ a_k \}_{k \in \mathbb{N}}$. The sequence $\{s_n\}_{n \in \mathbb{N}}$ of partial sums
is called a series.

\end{definition}

\vspace{0.4cm}

Remark: $s_n = s_{n-1} + a_n$.


\end{frame}%


\begin{frame}{\secname}

\frametitle{Series }


\begin{example}
Let us consider again the sequence $\{ a_k \}_{k \in \mathbb{N}}$ with $a_k = \frac{1}{k}$. Its partial sums are
$$
s_n =  \sum_{k=1}^{n} a_k.
$$
For instance, when $n=1,2,3$ we have:
\begin{eqnarray*}
s_1 &=& 1 \\
s_2 &=& 1 + \frac{1}{2} = \frac{3}{2}  \\
s_3 &=& 1 + \frac{1}{2} + \frac{1}{3} = s_2 + \frac{1}{3} = \frac{11}{6}.
\end{eqnarray*}


\end{example}

\end{frame}%


\end{document}
