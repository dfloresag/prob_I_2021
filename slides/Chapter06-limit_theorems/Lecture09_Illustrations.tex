\documentclass[notes=show,smaller]{beamer}\usepackage[]{graphicx}\usepackage[]{color}
% maxwidth is the original width if it is less than linewidth
% otherwise use linewidth (to make sure the graphics do not exceed the margin)
\makeatletter
\def\maxwidth{ %
  \ifdim\Gin@nat@width>\linewidth
    \linewidth
  \else
    \Gin@nat@width
  \fi
}
\makeatother

\definecolor{fgcolor}{rgb}{0.345, 0.345, 0.345}
\newcommand{\hlnum}[1]{\textcolor[rgb]{0.686,0.059,0.569}{#1}}%
\newcommand{\hlstr}[1]{\textcolor[rgb]{0.192,0.494,0.8}{#1}}%
\newcommand{\hlcom}[1]{\textcolor[rgb]{0.678,0.584,0.686}{\textit{#1}}}%
\newcommand{\hlopt}[1]{\textcolor[rgb]{0,0,0}{#1}}%
\newcommand{\hlstd}[1]{\textcolor[rgb]{0.345,0.345,0.345}{#1}}%
\newcommand{\hlkwa}[1]{\textcolor[rgb]{0.161,0.373,0.58}{\textbf{#1}}}%
\newcommand{\hlkwb}[1]{\textcolor[rgb]{0.69,0.353,0.396}{#1}}%
\newcommand{\hlkwc}[1]{\textcolor[rgb]{0.333,0.667,0.333}{#1}}%
\newcommand{\hlkwd}[1]{\textcolor[rgb]{0.737,0.353,0.396}{\textbf{#1}}}%
\let\hlipl\hlkwb

\usepackage{framed}
\makeatletter
\newenvironment{kframe}{%
 \def\at@end@of@kframe{}%
 \ifinner\ifhmode%
  \def\at@end@of@kframe{\end{minipage}}%
  \begin{minipage}{\columnwidth}%
 \fi\fi%
 \def\FrameCommand##1{\hskip\@totalleftmargin \hskip-\fboxsep
 \colorbox{shadecolor}{##1}\hskip-\fboxsep
     % There is no \\@totalrightmargin, so:
     \hskip-\linewidth \hskip-\@totalleftmargin \hskip\columnwidth}%
 \MakeFramed {\advance\hsize-\width
   \@totalleftmargin\z@ \linewidth\hsize
   \@setminipage}}%
 {\par\unskip\endMakeFramed%
 \at@end@of@kframe}
\makeatother

\definecolor{shadecolor}{rgb}{.97, .97, .97}
\definecolor{messagecolor}{rgb}{0, 0, 0}
\definecolor{warningcolor}{rgb}{1, 0, 1}
\definecolor{errorcolor}{rgb}{1, 0, 0}
\newenvironment{knitrout}{}{} % an empty environment to be redefined in TeX

\usepackage{alltt}
%%%%%%%%%%%%%%%%%%%%%%%%%%%%%%%%%%%%%%%%%%%%%%%%%%%%%%%%%%%%%%%%%%%%%%%%%%%%%%%%%%%%%%%%%%%%%%%%%%%%%%%%%%%%%%%%%%%%%%%%%%%%%%%%%%%%%%%%%%%%%%%%%%%%%%%%%%%%%%%%%%%%%%%%%%%%%%%%%%%%%%%%%%%%%%%%%%%%%%%%%%%%%%%%%%%%%%%%%%%%%%%%%%%%%%%%%%%%%%%%%%%%%%%%%%%%
\usepackage{amssymb}
\usepackage{amsmath}
\usepackage{graphicx}
\usepackage{hyperref}
\usepackage{multimedia}
\usepackage{epstopdf}
\usepackage{color}

\setcounter{MaxMatrixCols}{10}
\newtheorem{remark}{Remark}[section]
\newtheorem{proposition}{Proposition}[section]
\newtheorem{interpretation}{Interpretation}[section]
\newtheorem{goal}{Goal}[section]
\newtheorem{statement}{Statement}[section]
\newtheorem{aes}{Aim \& Scope}[section]
\newtheorem{exercise}{Exercise}[section]
\renewcommand{\Pr}{P}

\newcommand{\mbf}[1]{\mathbf{#1}}
\newcommand{\beq}{\begin{equation}}
\newcommand{\eeq}{\end{equation}}
\newcommand{\bea}{\begin{eqnarray}}
\newcommand{\eea}{\end{eqnarray}}
\newcommand{\ba}{\begin{array}}
\newcommand{\ea}{\end{array}}
\newcommand{\bi}{\begin{itemize}}
\newcommand{\ei}{\end{itemize}}
\newcommand{\ben}{\begin{enumerate}}
\newcommand{\een}{\end{enumerate}}
\newcommand{\nn}{\nonumber}

\newenvironment{stepenumerate}{\begin{enumerate}[<+->]}{\end{enumerate}}
\newenvironment{stepitemize}{\begin{itemize}[<+->]}{\end{itemize} }
\newenvironment{stepenumeratewithalert}{\begin{enumerate}[<+-| alert@+>]}{\end{enumerate}}
\newenvironment{stepitemizewithalert}{\begin{itemize}[<+-| alert@+>]}{\end{itemize} }
\usetheme{Madrid}


%%%%%%%%%%%%%%%%%%%%%%%%%%%%%%%%%%%%%%%%%%%%%%%%%%%%%%%%%%%%%%%%%%%%%%%%%%%%%%%
% GSEM COLORS
%%%%%%%%%%%%%%%%%%%%%%%%%%%%%%%%%%%%%%%%%%%%%%%%%%%%%%%%%%%%%%%%%%%%%%%%%%%%%%%
\definecolor{darkGSEM}{RGB}{70,95,127}
\definecolor{darkGSEM2}{RGB}{40,80,150}
\definecolor{GSEM}{RGB}{96,121,153} % GSEM 10% lighter

%%% Global colors
\setbeamercolor*{palette primary}{use=structure,fg=white,bg=darkGSEM}
\setbeamercolor*{palette quaternary}{use=structure,fg=white,bg=darkGSEM!90}
\setbeamercolor{frametitle}{fg=white,bg=GSEM!80}

%%% TOC colors
\setbeamercolor{section in toc}{fg=darkGSEM}

%%% itemize colors
\setbeamertemplate{itemize items}[circle]
\setbeamercolor{itemize item}{fg=darkGSEM2}
\setbeamercolor{itemize subitem}{fg=darkGSEM2}
\setbeamercolor{itemize subsubitem}{fg=darkGSEM2}


%%% enumerate colors
\setbeamercolor{item projected}{fg=white,bg=GSEM}
\setbeamertemplate{enumerate item}{\insertenumlabel.}
\setbeamercolor{enumerate item}{fg=darkGSEM2}
\setbeamercolor{enumerate subitem}{fg=darkGSEM2}
\setbeamercolor{enumerate subsubitem}{fg=darkGSEM2}


\AtBeginSection[]
{
  \begin{frame}{\secname}{\secname}
    \frametitle{Outline}
    \tableofcontents[currentsection]
  \end{frame}
}

%%%%%%%%%%%%%%%%%%%%%%%%%%%%%%%%%%%%%%%%%%%%%%%%%%%%%%%%%%%%%%%%%%%%%%%%%%%%%%%%
% R Chunk Options and Packages
%%%%%%%%%%%%%%%%%%%%%%%%%%%%%%%%%%%%%%%%%%%%%%%%%%%%%%%%%%%%%%%%%%%%%%%%%%%%%%%%


%%%%%%%%%%%%%%%%%%%%%%%%%%%%%%%%%%%%%%%%%%%%%%%%%%%%%%%%%%%%%%%%%%%%%%%%%%%%%%%%
\title[S110015]{Probability 1}
\subtitle{Lecture 09 : Illustration of the Central Limit Theorem}
\author[Flores-Agreda, La Vecchia]{Dr. Daniel Flores-Agreda, \\[0.5em] \tiny{(based on the notes of Prof. Davide La Vecchia)}}
\date{Spring Semester 2021}
\IfFileExists{upquote.sty}{\usepackage{upquote}}{}
\begin{document}

\begin{frame}
  \titlepage
\end{frame}


% \begin{frame}{Objectives}
%   \begin{itemize}
%   \item . \bigskip
%   \item .
%   \end{itemize}
% \end{frame}

% \begin{frame}{Outline}
%   \tableofcontents
% \end{frame}

%
% %%%%%%%%%%%%%%%%%%%%%%%%%%%%%%%%%%%%%%%%%%%%%%%%%%%%%%%%%%%%%%%%%%%%%%%%%%%%%%%%
% \section{Weak Law of Large Numbers}
% %%%%%%%%%%%%%%%%%%%%%%%%%%%%%%%%%%%%%%%%%%%%%%%%%%%%%%%%%%%%%%%%%%%%%%%%%%%%%%%%
%
% \begin{frame}{\secname}
% \framesubtitle{Reminder}
%   \begin{proposition}[Simplified WLLN]
%    Let $X_{1},X_{2},\dots,X_{n},\dots$ be a sequence of \textit{i.i.d.} random variables with common probability distribution $F_X(x)$, and let $Y=h(X)$ be such that
%   \begin{eqnarray*}
%   E[Y]=E\left[ h(X)\right]  &=&\mu_Y  \\
%   Var(Y)=Var\left( h(X)\right)  &=&\sigma_Y ^{2}<\infty\,.
%   \end{eqnarray*}%
%   Set
%   $$
%   \overline{Y}_n=\frac{1}{n}\sum_{s=1}^nY_s\quad\text{where}\quad Y_s=h(X_s)\,,\quad s=1,\ldots,n\,.
%   $$
%   Then, for any two numbers $\varepsilon$ and $\delta$ satisfying $\varepsilon>0$ and $0<\delta<1$
%   $$
%   \Pr \left( \left\vert \overline{Y}_{n}-\mu_Y \right\vert<\varepsilon \right)\geq 1-\delta
%   $$
%   for all $n>\sigma_Y^2/(\varepsilon^2\delta)$.
%
%   Choosing both $\varepsilon$ and $\delta$ to be arbitrarily small implies that $p\lim_{n\rightarrow\infty}(\overline{Y}_{n}-\mu_Y)=0$, or equivalently
%   $\overline{Y}_{n}\overset{p}{\rightarrow }\mu_Y$.
%   \end{proposition}
% \end{frame}
%
% \begin{frame}{\secname}
%
% \end{frame}

%%%%%%%%%%%%%%%%%%%%%%%%%%%%%%%%%%%%%%%%%%%%%%%%%%%%%%%%%%%%%%%%%%%%%%%%%%%%%%%%
\section{Central Limit Theorem}
%%%%%%%%%%%%%%%%%%%%%%%%%%%%%%%%%%%%%%%%%%%%%%%%%%%%%%%%%%%%%%%%%%%%%%%%%%%%%%%%

\begin{frame}{\secname}
  \framesubtitle{Simplified Version}
\begin{theorem} Let $X_{1},X_{2},...,X_{n},...$ be a sequence of \textit{i.i.d.} random variables and let $Y=h(X)$ be such that
  \begin{eqnarray*}
  E[Y]=E\left[ h(X)\right]  &=&\mu_Y  \\
  Var(Y)=Var\left( h(X)\right)  &=&\sigma_Y ^{2}<\infty\,.
  \end{eqnarray*}%
  Set
  $$
  \overline{Y}_n=\frac{1}{n}\sum_{s=1}^nY_s\quad\text{where}\quad Y_s=h(X_s)\,,\quad s=1,\ldots,n\,.
  $$
  Then (under quite general regularity conditions)%
  \begin{equation*}
  \frac{\sqrt{n}\left( \overline{Y}_{n}-\mu_Y \right) }{\sigma_Y }\overset{D}{%
  \rightarrow }N\left( 0,1\right)
  {\color{red}
  \Leftrightarrow
  P\left(\frac{\sqrt{n}\left( \overline{Y}_{n}-\mu_Y \right) }{\sigma_Y} \leq x \right)
  \underset{n\rightarrow \infty}{\longrightarrow} \Phi(x)}
  \end{equation*}
  \end{theorem}
\end{frame}



\begin{frame}{\secname}
\framesubtitle{Implications}
\end{frame}

\begin{frame}{\secname}
  \begin{example}[Ross, Example 3e]
  \begin{footnotesize}
  An instructor has 50 exams that will be graded in sequence.

  \medskip

  The times required to grade the 50 exams are independent, with a common distribution that has mean 20 minutes and standard deviation of 4 minutes.

  \medskip

  \emph{Approximate the probability that the instructor will grade at least 25 of the exams in the first 450 minutes of work.}
  \end{footnotesize}
  \end{example}
\end{frame}

\begin{frame}{\secname}
  \begin{example}[Ross, Example 3e]
  \begin{footnotesize}

\vspace{7cm}

  \end{footnotesize}
  \end{example}
\end{frame}




\end{document}
