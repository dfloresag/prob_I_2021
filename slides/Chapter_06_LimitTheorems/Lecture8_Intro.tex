\documentclass[notes=show,smaller,handout]{beamer}
%%%%%%%%%%%%%%%%%%%%%%%%%%%%%%%%%%%%%%%%%%%%%%%%%%%%%%%%%%%%%%%%%%%%%%%%%%%%%%%%%%%%%%%%%%%%%%%%%%%%%%%%%%%%%%%%%%%%%%%%%%%%%%%%%%%%%%%%%%%%%%%%%%%%%%%%%%%%%%%%%%%%%%%%%%%%%%%%%%%%%%%%%%%%%%%%%%%%%%%%%%%%%%%%%%%%%%%%%%%%%%%%%%%%%%%%%%%%%%%%%%%%%%%%%%%%
\usepackage{amssymb}
\usepackage{amsmath}
\usepackage{graphicx}
\usepackage{hyperref}
\usepackage{multimedia}
\usepackage{epstopdf}
\usepackage{color}

\setcounter{MaxMatrixCols}{10}
\newtheorem{remark}{Remark}[section]
\newtheorem{proposition}{Proposition}[section]
\newtheorem{interpretation}{Interpretation}[section]
\newtheorem{goal}{Goal}[section]
\newtheorem{statement}{Statement}[section]
\newtheorem{aes}{Aim \& Scope}[section]
\newtheorem{exercise}{Exercise}[section]
\renewcommand{\Pr}{P}

\newcommand{\mbf}[1]{\mathbf{#1}}
\newcommand{\beq}{\begin{equation}}
\newcommand{\eeq}{\end{equation}}
\newcommand{\bea}{\begin{eqnarray}}
\newcommand{\eea}{\end{eqnarray}}
\newcommand{\ba}{\begin{array}}
\newcommand{\ea}{\end{array}}
\newcommand{\bi}{\begin{itemize}}
\newcommand{\ei}{\end{itemize}}
\newcommand{\ben}{\begin{enumerate}}
\newcommand{\een}{\end{enumerate}}
\newcommand{\nn}{\nonumber}

\newenvironment{stepenumerate}{\begin{enumerate}[<+->]}{\end{enumerate}}
\newenvironment{stepitemize}{\begin{itemize}[<+->]}{\end{itemize} }
\newenvironment{stepenumeratewithalert}{\begin{enumerate}[<+-| alert@+>]}{\end{enumerate}}
\newenvironment{stepitemizewithalert}{\begin{itemize}[<+-| alert@+>]}{\end{itemize} }
\usetheme{Madrid}



\begin{document}


\title[S110015]{Probability I}
\subtitle{Lecture 8 (additional material)}
\author[La Vecchia]{Davide La Vecchia}
\date{Spring Semester 2020}
\maketitle


\begin{frame}

\frametitle{Sequences of real numbers }

\begin{definition}
 A sequence is an ordered list of real numbers
of the form%
\begin{equation*}
a_{1},a\,_{2},...,a_{n},...
\end{equation*}
where each natural number $n \in \mathbb{N}$ corresponds exactly to a real number $a_n \in \mathbb{R}$. A sequence is denoted
by $\{a_n \}_{n \in \mathbb{N}}$, where $n$ is called the index of the sequence and $a_n$ is its n-th term.
\end{definition}

\vspace{0.4cm} 

Remark: the sequence can contain infinite terms...

\begin{example}
The list of numbers 
$$
\Big\{1,\frac{1}{2},\frac{1}{3},\frac{1}{4},... \Big \} 
$$
is a sequence, where each natural number corresponds the real number $a_n = \frac{1}{n}$.
\end{example}



\end{frame}



\begin{frame}

\frametitle{Limit of a sequences of real numbers }

\begin{definition}
A real number  $a\in \mathbb{R}$ (a is a finite number) is called the limit of a sequence $\{a_n \}_{n \in \mathbb{N}}$ if, for any $\epsilon >0$, a natural number $n(\epsilon) \in \mathbb{N}$ exists such that
\begin{equation} \label{Eq. lim}
\vert a_{n}-a \vert < \epsilon \quad  \text{for all \ \ } n\geq n(\epsilon).
\end{equation}
If for a given sequence $\{a_n \}_{n \in \mathbb{N}}$ the real number $a$ satisfies (\ref{Eq. lim}), then we write
$$
a = \lim_{n\to \infty} a_n.
$$
\end{definition}

\begin{example}
The sequence $\{ a_n \}_{n \in \mathbb{N}}$
with $a_n = \frac{1}{n}$, converges to zero.
\end{example}

Remark: loosely speaking,  Eq. (\ref{Eq. lim}) states that, for $n \geq n(\epsilon)$, $a_n$ is \textbf{always} close to $a$ (or equivalently, the difference in absolute value between $a_n$ and $a$ is \textbf{never} large). 

\end{frame}%

\begin{frame}

\frametitle{Series }


\begin{definition}
Let $\{ a_k \}_{k \in \mathbb{N}}$ be a  sequence. The sum of the first $n$ terms of  $\{ a_k \}_{k \in \mathbb{N}}$:
$$
s_n = \sum_{k=1}^{n} a_k = a_1 + a_2 +...+a_n
$$
is called the $n$-th partial sum of $\{ a_k \}_{k \in \mathbb{N}}$. The sequence $\{s_n\}_{n \in \mathbb{N}}$ of partial sums
is called a series. 

\end{definition}

\vspace{0.4cm} 

Remark: $s_n = s_{n-1} + a_n$.


\end{frame}%


\begin{frame}

\frametitle{Series }


\begin{example}
Let us consider again the sequence $\{ a_k \}_{k \in \mathbb{N}}$ with $a_k = \frac{1}{k}$. Its partial sums are 
$$
s_n =  \sum_{k=1}^{n} a_k.
$$
For instance, when $n=1,2,3$ we have:
\begin{eqnarray*}
s_1 &=& 1 \\
s_2 &=& 1 + \frac{1}{2} = \frac{3}{2}  \\
s_3 &=& 1 + \frac{1}{2} + \frac{1}{3} = s_2 + \frac{1}{3} = \frac{11}{6}.
\end{eqnarray*}


\end{example}

\end{frame}%


\end{document}
